\documentclass[11pt]{article}

\usepackage{listings}
\lstset{basicstyle=\ttfamily, tabsize=4, columns=flexible, breaklines=true, stepnumber=1, numberstyle=\tiny, numbersep=6pt, xleftmargin=1.8em}
\usepackage[dutch]{babel}
\usepackage{parskip}


\begin{document}

\author{Roan Kattouw and Jan Paul Posma}
\date{\today}
\title{Operating Systems practicum 2}

\maketitle

\section*{Opgave 1}
Op regel 7 wordt \verb+PAGESIZE+ gedefini\"eerd als de grootte van een page in het doel-OS.
Voor Linux op een x32 is dat 4096 bytes. Op regels 11-14 wordt een functie gedefini\"eerd die
\verb+"Memory was protected"+ print en het eerder beschermde geheugenblok vrijgeeft. Deze functie
wordt door regel 20 als signal handler voor het \verb+SIGSEGV+ signaal ge\"installeerd.

Het programma alloceert eerst een blok geheugen dat aligned is op een veelvoud van \verb+PAGESIZE+;
dit gebeurt op regel 25. Regels 31 en 32 schrijven nu naar dat blok geheugen; dit gaat goed
omdat het geheugen niet beschermd is. Regel 35 beschermt nu het blok en maakt het read-only.
Regel 40 leest een getal uit het blok; dit mag, want het blok is alleen tegen schrijven
beschermd, dus dit drukt \verb+val = 42+ af.

Regel 42 probeert nu het getal 84 in \verb+arr[666]+ te zetten. Dit mag niet, want het blok
is beschermd tegen schrijven, dus er wordt een \verb+SIGSEGV+ signaal verzonden. Dit signaal
wordt afgevangen en de handler op regels 11-14 wordt aangeroepen. Deze drukt \verb+"Memory was protected"+
af en maakt het geheugen schrijfbaar. Nu de handler klaar is, wordt regel 42 opnieuw uitgevoerd.
De tweede keer slaagt de schrijfactie wel, omdat de handler de bescherming heeft weggehaald.
Regel 44 print nu \verb+val = 84+.

\section*{Opgave 2}
Onze shared pages implementatie gebruikt twee pipes (\'e\'en voor communicatie van kind naar ouder en \'e\'en voor andersom;
dezelfde pipe voor beide gebruiken zorgt voor bugs) om pages heen en weer te sturen, en
\verb+SIGUSR1+ om te signaleren dat het andere proces de page wil hebben. De processen
doen ping-pong met busy waiting. De \verb+turn+ variabele is een \verb+int+ die de eerste
4 bytes van de page in beslag neemt.

\subsection*{Initialisatie}
\begin{itemize}
\item Er wordt een buffer van \'e\'en page (4096 bytes), aligned op een meervoud van 4096 bytes, gealloceerd
\item Er worden handlers ge\"installeerd voor \verb+SIGSEGV+ en \verb+SIGUSR1+
\item Er worden twee pipes geopend
\item \verb+fork()+ wordt aangeroepen
\item Beide processen sluiten de uiteinden van de pipes die ze niet gebruiken
\item Het kindproces beschermt zijn page tegen lezen en schrijven, het ouderproces staat beide toe
\end{itemize}

\subsection*{Lezen/schrijven naar de gedeelde page}
Op elk moment is slechts \'e\'en proces de 'eigenaar' van de page. In dit proces is toegang
tot de page niet beschermd door \verb+mprotect()+, in het andere proces wel. Als het proces
dat niet de eigenaar is, probeert te lezen uit of schrijven naar de gedeelde page, dan gebeurt
er het volgende:
\begin{itemize}
\item Doordat de page beschermd is tegen lezen en schrijven, wordt er een \verb+SIGSEGV+ signaal verzonden
\item De functie \verb+segvHandler()+ vangt dit signaal af
\item De handler controleert dat het adres waarvoor de \verb+SIGSEGV+ verzonden is. Deze informatie staat in de \verb+siginfo_t+ datastructuur die meegegeven wordt omdat de handler is ge\"installeerd met de \verb+SA_SIGINFO+ flag
\item Als de segfault van buiten de page komt, wordt de standaardactie voor \verb+SIGSEGV+ hersteld. Dit leidt ertoe dat het proces sterft met een segmentation fault
\item De segfault komt van binnen de page, dus de page moet worden overgedragen vanuit het andere proces. Er wordt een \verb+SIGUSR1+ signaal naar het andere proces gestuurd om aan te geven dat de page van eigenaar moet wisselen
\item Het andere proces ontvangt het \verb+SIGUSR1+ signaal, schrijft de inhoud van de page naar de pipe, en beschermt de page tegen lezen en schrijven
\item Het ontvangende proces haalt de bescherming van de page weg, en leest de inhoud van de page in van de pipe
\item De handler eindigt. De instructie die de segfault veroorzaakte, wordt herstart, en slaagt nu wel omdat de page niet meer beschermd is
\end{itemize}

\subsection*{Aan het einde}
Als een proces klaar is met pingpongen maar het andere nog niet, mag het niet zomaar termineren. Het andere proces zou dan immers
vast kunnen komen te zitten, omdat het niet in bezit is van de page en de eigenaar al getermineerd is. Om deze situatie te voorkomen,
laten we het ouderproces, dat altijd als eerste klaar is omdat het begint, de page onvoorwaardelijk naar het andere proces sturen.
Als het andere proces er op dat moment niet op zit te wachten, is dat niet erg: de page wordt dan gebufferd in de pipe. Het ouderproces
laat vervolgens alle toekomstige \verb+SIGUSR1+ signalen negeren en roept \verb+waitpid()+ aan om te wachten op het kindproces. Het kind kan nu bezit
nemen van de page en zijn werk afmaken zonder dat de ouder nog iets hoeft te doen.

\subsection*{Race conditions voorkomen}
De enige staat die wordt bijgehouden in de processen, is de beschermingsstatus van de lokale kopie van de page. Deze staat wordt
alleen door interactie vanuit de signal handlers veranderd, dus het is vrij makkelijk om te zorgen dat hij consistent blijft. Het
enige wat we hoeven te doen is zorgen dat de signal handler voor \verb+SIGSEGV+ niet onderbroken kan worden door de handler voor
\verb+SIGUSR1+ (en andersom, al is dat zeer onwaarschijnlijk). Dit doen we door de handler voor het ene signaal te installeren met
het andere signaal in de signal mask. Als er dan een \verb+SIGUSR1+ ontvangen wordt terwijl de handler voor \verb+SIGSEGV+ actief
is, wordt de \verb+SIGUSR1+ geblokkeerd en pas afgeleverd (i.e. zijn handler wordt pas aangeroepen) als de \verb+SIGSEGV+ handler
klaar is.

\section*{Opgave 3}
Als extra administratie houdt ieder proces in de variabele \verb+dirty+ de status van de page bij:
\begin{itemize}
\item \verb+dirty=0+: De page is up-to-date in beide processen. De page is beschermd tegen schrijven in beide processen.
\item \verb+dirty=1+: De page is veranderd, het huidige proces heeft de nieuwste versie. De page is niet beschermd in het huidige proces en beschermd tegen lezen en schrijven in het andere proces.
\item \verb+dirty=2+: De page is veranderd, het andere proces heeft de nieuwste versie. De page is beschermd tegen lezen en schrijven in het huidige proces en niet beschermd in het andere proces.
\end{itemize}

Initi\"eel is de page up-to-date in beide processen, dus \verb+dirty=0+. In deze situatie kunnen
beide processen hun exemplaar zonder belemmering lezen, en is de page alleen tegen schrijven beschermd.
Als een proces nu naar de page probeert te schrijven, haalt de \verb+SIGSEGV+ handler de schrijfbescherming
weg, zet \verb+dirty=1+ en stuurt \verb+SIGUSR2+ naar het andere proces. Het andere proces ontvangt dit signaal,
zet \verb+dirty=2+ en beschermt zijn exemplaar tegen lezen en schrijven. Als nu het andere proces probeert
te lezen of schrijven, wordt de page uitgewisseld net zoals in opgave 2, en zetten beide processen \verb+dirty=0+
en maken hun page read-only.

\subsection*{Race conditions, with a vengance}
Het blokkeren van signalen in de handlers is nog steeds nodig om een inconsistente staat te voorkomen, maar het
is niet voldoende. Er kan nog steeds een race condition optreden als beide processen in de staat \verb+dirty=0+
zitten en ze allebei tegelijk proberen te schrijven. Als het eerste proces \verb+SIGUSR2+ stuurt om het tweede
te laten weten dat het de page veranderd heeft, maar het tweede proces al in de \verb+if(dirty == 0)+ branch
zit, ontstaat een inconsistente staat. Beide processen komen eerst in de staat \verb+dirty=1+, laten
allebei de schrijfactie toe, ontvangen vervolgens elkaars \verb+SIGUSR2+ signalen en komen in de staat
\verb+dirty=2+. Hierna wordt bij de eerstvolgende lees- of schrijfactie de staat weer consistent. Het is
dan wel zo dat \'e\'en van de conflicterende schrijfacties die de race condition veroorzaakte, verloren gaat.

Dit is op te lossen door te voorkomen dat de twee processen tegelijk de \verb+SIGSEGV+ handler ingaan.
Dit doen we door middel van een \verb+flock()+ op een van de twee pipes aan het begin van de handler.
Merkwaardig hierbij is dat beide processen een exclusief lock op hetzelfde uiteinde van de pipe laten doen
niet werkt (ze krijgen het allebei), maar als de processen ieder op een ander uiteinde van de pipe een lock
vragen (i.e. de ouder lockt \verb+fds[0]+ en het kind lockt \verb+fds[1]+), is er wel wederzijdse uitsluiting.
Het proces wat daar als tweede aankomt zal blokkeren tot het eerste proces aan het eind van de handler
komt en de lock vrijgeeft. Het eerste proces zal in de tussentijd waarschijnlijk een \verb+SIGUSR1+ sturen;
het tweede proces ontvangt en verwerkt dit en hervat automatisch het wachten op de lock. Als het lock vrijkomt
en het tweede proces het verkrijgt, is \verb+dirty+ in de beide processen alweer in een consistente staat,
en ontstaat er geen race condition.

Bovendien is het zo dat in het specifieke geval van ons busy waiting programma deze race condition niet kan
optreden: de enige assignment aan de shared variable \verb+turn+ vindt plaats bij het verlaten van de kritieke
sectie, en daar kunnen de processen dus niet tegelijk zijn.

\subsection*{Benchmarks}
Deze optimalisatie vermindert het verkeer over de pipe, maar maakt opmerkelijk genoeg het programma juist langzamer.
Op mijn machine duurden een miljoen iteraties met de code van opgave 2 17,7 seconden, en met de code van deze
opgave 49,52 seconden. Het lijkt erop dat de extra overhead van \verb+flock()+ en \verb+SIGUSR2+, die nodig
is voor het bijhouden en consistent houden van de staat, meer kost dan er bespaard wordt aan goedkope I/O
over een pipe. Wellicht dat als de communicatie over een trager medium zoals een netwerk gaat, er wel winst is.

\end{document}