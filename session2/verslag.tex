\documentclass[11pt]{article}

\usepackage{listings}
\lstset{basicstyle=\ttfamily, tabsize=4, columns=flexible, breaklines=true, stepnumber=1, numberstyle=\tiny, numbersep=6pt, xleftmargin=1.8em}
\usepackage[dutch]{babel}
\usepackage{parskip}


\begin{document}

\author{Roan Kattouw and Jan Paul Posma}
\date{\today}
\title{Operating Systems practicum 2}

\maketitle

\section*{iwish}

\section*{UTCtime}
Allereerst hebben we de system call gemaakt en daarna de library call.


% LET OP: de Makefile bestanden zelf (niet Makefile.in) en /usr/src/lib/posix/Makefile.in missen nog.. Moet ik
% nog op het practicumsysteem ophalen, en dit wil niet via SSH.. :(
\subsection*{System call}
De system call hebben we bij de andere time functies gezet in \verb+/usr/src/servers/pm/time.c+.
Aangezien het na\"ief is om te denken dat we simpelweg alle oude leap-seconden van het huidige
aantal kunnen aftrekken, kijken we daadwerkelijk naar de exacte tijd. Het is namelijk ook mogelijk
dat er bijvoorbeeld een oude gebeurtenis wordt nagebootst in een Virtual Machine, met de klok
dus teruggezet. In veruit de meeste gevallen zal echter simpelweg het getal 24 (aantal leap-seconden tot
nu toe) moeten worden afgetrokken. Daarom controleren we eerst of we dit kunnen doen; dat is sneller.

\begin{verbatim}
	#define LEAP(x) if(rt>=x) { rt--; }

	PUBLIC int do_utctime()
	{
	  clock_t uptime, boottime;
	  time_t rt;
	  int s;
	
	  if ( (s=getuptime2(&uptime, &boottime)) != OK)
			panic(__FILE__, "do_utctime couldn't get uptime", s);
	
	  rt = (time_t) (boottime + (uptime/system_hz));
	  mp->mp_reply.reply_utime = (uptime%system_hz)*1000000/system_hz;
	  
	/* Correct for leap seconds.
	 */
	  LEAP(78796800); LEAP(94694400); LEAP(126230400); LEAP(157766400); LEAP(189302400);
	  LEAP(220924800); LEAP(252460800); LEAP(283996800); LEAP(315532800); LEAP(362793600); 
	  LEAP(394329600); LEAP(425865600); LEAP(489024000); LEAP(567993600); LEAP(631152000); 
	  LEAP(662688000); LEAP(709948800); LEAP(741484800); LEAP(773020800); LEAP(820454400); 
	  LEAP(867715200); LEAP(915148800); LEAP(1136073600); LEAP(1230768000);
	  
	  mp->mp_reply.reply_time = rt;
	  
	  return(OK);
	}
\end{verbatim}

Verder moesten nog enkele bestanden worden aangepast. In \verb+/usr/src/servers/pm/proto.h+ hebben
we het functieprototype toegevoegd:

\begin{verbatim}
_PROTOTYPE( int do_utctime, (void) );
\end{verbatim}

Verder hebben we voor de unused positie 69 gekozen voor onze system call. In \verb+/usr/src/include/minix/callnr.h+
zorgt dit voor:

\begin{verbatim}
#define UTCTIME 69
\end{verbatim}

Daarnaast hebben we in \verb+/usr/src/servers/pm/table.c+ op positie 69 \verb+no_sys+ vervangen
door \verb+do_utctime+.

\subsection*{Library call}

Het maken van een library call is relatief eenvoudig. We zijn wel in twee valkuilen gelopen. Ten eerste
moet er ook een \verb+utctime.s+ bestand worden aangemaakt, wat niet overal vermeld wordt. Daarnaast
werkt het ook niet wanneer het testprogramma gecompileerd wordt met \verb+gcc+ in plaats van \verb+cc+.

Allereerst hebben we in \verb+/usr/src/lib/posix/_utctime.c+ een wrapper gemaakt, afgekeken van de
\verb+time+ system call:

\begin{verbatim}
#include <lib.h>
#define time _time
#include <time.h>

PUBLIC time_t utctime(tp)
time_t *tp;
{
  message m;
  if (_syscall(MM, UTCTIME, &m) < 0) return( (time_t) -1);
  if (tp != (time_t *) 0) *tp = m.m2_l1;
  return(m.m2_l1);
}
\end{verbatim}

Hierbij moest ook \verb+_utctime.c+ in \verb+/usr/src/lib/posix/Makefile.in+ worden toegevoegd.
De functiedefinitie hebben we in \verb+/usr/src/include/time.h+ geplaatst:

\begin{verbatim}
_PROTOTYPE( time_t utctime, (time_t *_timeptr) );
\end{verbatim}

Tenslotte moest de eerdergenoemde \verb+utctime.s+ worden aangemaakt, namelijk op
\verb+/usr/src/lib/syscall/utctime.s+:

\begin{verbatim}
.sect .text
.extern __utctime
.define _utctime
.align 2

_utctime:
	jmp __utctime
\end{verbatim}

Natuurlijk moest ook \verb+utctime.s+ worden toegevoegd aan \verb+/usr/src/lib/syscall/Makefile.in+.

Ten slotte een eenvoudig test-programma, \verb+test.c+:

\begin{verbatim}
#include<time.h>
#include<stdio.h>

int main()
{
  printf("time    = %u\n", time(0));
  printf("utctime = %u\n", utctime(0));
  return 0;
}
\end{verbatim}

Dit laat inderdaad de gewenste uitvoer zien:

\begin{verbatim}
time    = 1307460054
utctime = 1307460030
\end{verbatim}

\end{document}
